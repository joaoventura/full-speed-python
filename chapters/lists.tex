\chapter{Lists}\label{lists}

Python lists are data structures that group sequences of elements. Lists can have elements of several types and you can also mix different types within the same list although all elements are usually of the same datatype. 

Lists are created using square brackets and the elements separated by commas. The elements in a list can be accessed by their positions where 0 is the index of the first element:

\begin{lstlisting}
>>> l = [1, 2, 3, 4, 5]
>>> l[0]
1
>>> l[1]
2
\end{lstlisting}

Can you access the number 4 in the previous list?

Sometimes you want just a small portion of a list, a sublist. Sublists can be retrieved using a technique called \textit{slicing}, which consists on defining the start and end indexes:

\begin{lstlisting}
>>> l = ['a', 'b', 'c', 'd', 'e']
>>> l[1:3]
['b', 'c']
\end{lstlisting}

Finally, arithmetic with lists is also possible, like adding two lists together or repeating the contents of a list.

\begin{lstlisting}
>>> [1,2] + [3,4]
[1, 2, 3, 4]
>>> [1,2] * 2
[1, 2, 1, 2]
\end{lstlisting}


\section{Exercises with lists}

Create a list named "l" with the following values ([1, 4, 9, 10, 23]). Using the Python documentation about lists (\url{https://docs.python.org/3.5/tutorial/introduction.html#lists}) solve the following exercises:

\begin{enumerate}

\item Using list slicing get the sublists [4, 9] and [10, 23].

\item Append the value 90 to the end of the list "l". Check the difference between list concatenation and the "append" method.

\item Calculate the average value of all values on the list. You can use the "sum" and "len" functions.

\item Remove the sublist [4, 9].

\end{enumerate}
